% Template for PLoS
% Version 1.0 January 2009
%
% To compile to pdf, run:
% latex plos.template
% bibtex plos.template
% latex plos.template
% latex plos.template
% dvipdf plos.template

\documentclass[10pt]{article}

% amsmath package, useful for mathematical formulas
\usepackage{amsmath}
% amssymb package, useful for mathematical symbols
\usepackage{amssymb}

% graphicx package, useful for including eps and pdf graphics
% include graphics with the command \includegraphics
\usepackage{graphicx}

% cite package, to clean up citations in the main text. Do not remove.
\usepackage{cite}

\usepackage{color} 

% Use doublespacing - comment out for single spacing
%\usepackage{setspace} 
%\doublespacing


% Text layout
\topmargin 0.0cm
\oddsidemargin 0.5cm
\evensidemargin 0.5cm
\textwidth 16cm 
\textheight 21cm

% Bold the 'Figure #' in the caption and separate it with a period
% Captions will be left justified
\usepackage[labelfont=bf,labelsep=period,justification=raggedright]{caption}

% Use the PLoS provided bibtex style
\bibliographystyle{plos2009}

% Remove brackets from numbering in List of References
\makeatletter
\renewcommand{\@biblabel}[1]{\quad#1.}
\makeatother


% Leave date blank
\date{}

\pagestyle{myheadings}
%% ** EDIT HERE **


%% ** EDIT HERE **
%% PLEASE INCLUDE ALL MACROS BELOW

%% END MACROS SECTION

\begin{document}

% Title must be 150 characters or less
\begin{flushleft}
{\Large
\textbf{A Reference-Free Algorithm for Computational Normalization of Shotgun Sequencing Data}
}
% Insert Author names, affiliations and corresponding author email.
\\
C. Titus Brown$^{1,2,\ast}$,
Chris Welcher$^{1}$,
\\
\bf{1} Computer Science and Engineering, Michigan State University,
East Lansing, MI, USA
\\
\bf{2} Microbiology and Molecular Genetics, Michigan State University,
East Lansing, MI, USA
\\
$\ast$ E-mail: ctb@msu.edu
\end{flushleft}

% Please keep the abstract between 250 and 300 words
\section*{Abstract}


% Please keep the Author Summary between 150 and 200 words
% Use first person. PLoS ONE authors please skip this step. 
% Author Summary not valid for PLoS ONE submissions.   
\section*{Author Summary}

\section*{Introduction}

% Results and Discussion can be combined.
\section*{Results}

\subsection*{Assembling Massive mRNAseq}
Stats on lamprey data set
Lack of reliable reference (sorry Weiming?)
"We can do it!"
Preprocessing info
Memory, cpu time, etc stats


\subsection*{Whole transcriptome assembly yields novel transcripts}


\subsection*{}

\subsection*{}

\subsection*{Digital normalization scales assembly of transcriptomes}

%%%%%%%%%%%

\section*{Discussion}

\subsection*{Digital normalization dramatically scales {\em de novo} assembly}


\subsection*{A general strategy for normalizing coverage}


\subsection*{Digital normalization drops terminal k-mers and removes isoforms}

\subsection*{Applying assembly algorithms to digitally normalized data}


\subsection*{Conclusions}






\subsection*{Digital normalization is widely applicable and computationally convenient}
%


% You may title this section "Methods" or "Models". 
% "Models" is not a valid title for PLoS ONE authors. However, PLoS ONE
% authors may use "Analysis" 
\section*{Methods}

All links below are available electronically through
ged.msu.edu/papers/2012-diginorm/, in addition to the
archival locations provided.

\subsection*{Data sets}

The {\em E. coli}, {\em S. aureus}, and {\em Deltaproteobacteria} data
sets were taken from Chitsaz et al. \cite{pubmed21926975}, and
downloaded from bix.ucsd.edu/projects/singlecell/.  The
mouse data set was published by Grabherr et al. \cite{pubmed21572440}
and downloaded from trinityrnaseq.sf.net/.  All data sets
were used without modification.
The complete assemblies, both pre- and post-normalization, for the
{\em E. coli}, {\em S. aureus}, the uncultured {\em
  Deltaproteobacteria}, mouse, and yeast data sets are available from
ged.msu.edu/papers/2012-diginorm/.

The simulated genome and transcriptome were generated from a uniform
AT/CG distribution.  The genome consisted of a single chromosome
400,000 bases in length, while the transcriptome consisted of 100
transcripts of length 500.  100-base reads were generated uniformly
from the genome to an estimated coverage of 200x, with a random 1\%
per-base error.  For the transcriptome, 1 million reads of length 100
were generated from the transcriptome at relative expression levels of
10, 100, and 1000, with transcripts assigned randomly with equal
probability to each expression group; these reads also had a 1\%
per-base error.

\subsection*{Scripts and software}

All simulated data sets and all analysis summaries were generated by
Python scripts, which are available at
github.com/ctb/2012-paper-diginorm/.  Digital normalization and k-mer
analyses were performed with the khmer software package, written in
C++ and Python, available at github.com/ctb/khmer/, tag
'2012-paper-diginorm'.  khmer also relies on the screed package for
loading sequences, available at github.com/ctb/screed/, tag
'2012-paper-diginorm'.  khmer and screed are Copyright (c) 2010
Michigan State University, and are free software available for
distribution, modification, and redistribution under the BSD license.

Mapping was performed with bowtie v0.12.7 \cite{pubmed19261174}.
Genome assembly was done with velvet 1.2.01 \cite{pubmed18349386}.
Transcriptome assembly was done with velvet 1.1.05/oases 0.1.22 and
Trinity, head of branch on 2011.10.29.
Graphs and correlation coefficients were generated using matplotlib
v1.1.0, numpy v1.7, and ipython notebook v0.12 \cite{ipython}.  The
ipython notebook file and data analysis scripts necessary to generate
the figures are available at 
  github.com/ctb/2012-paper-diginorm/.

%%http://public.ged.msu.edu.s3-website-us-east-1.amazonaws.com/molgula/

\subsection*{Analysis parameters}

The khmer software uses a CountMin Sketch data structure to count
k-mers, which requires a fixed memory allocation
\cite{countminsketch}.  In all cases the memory usage was fixed such
that the calculated false positive rate was below 0.01.  By default k
was set to 20.

Genome and transcriptome coverage was calculated by mapping all reads
to the reference with bowtie ({\tt -a --best --strata}) and then
computing the per-base coverage in the reference.  Read coverage was
computed by then averaging the per-base reference coverage for each
position in the mapped read; where reads were mapped to multiple
locations, a reference location was chosen randomly for computing
coverage.  Median k-mer counts were computed with khmer as described
in the text.  Artificially high counts resulting from long stretches
of Ns were removed after the analysis.
Correlations between median k-mer counts and mapping coverage were
computed using numpy.corrcoef; see calc-r2.py script.

\subsection*{Normalization and assembly parameters}

For Table \ref{tab:dngenome}, the assembly k parameter for Velvet was
k=45 for {\em E. coli}; k=41 for {\em S. aureus} single cell; and k=39
for {\em Deltaproteobacteria} single cell.  Digital normalization
on the three bacterial samples was done with {\tt -N 4 -x 2.5e8 -k 20},
consuming 1gb of memory.  Post-normalization k parameters for Velvet
assemblies were k=37 for {\em E. coli}, k=27 for {\em S. aureus}, and k=27 for {\em Deltaproteobacteria}.
For Table \ref{tab:dntrans}, the assembly k parameter for Oases was k=21 for yeast
and k=23 for mouse.  Digital normalization on both mRNAseq samples was done
with {\tt -N 4 -x 2e9 -k 20}, consuming 8gb of memory.  Assembly of the
single-pass normalized mRNAseq was done with Oases at k=21 (yeast) and k=19
(mouse).

\subsection*{Assembly overlap and analysis}

Assembly overlap was computed by first using NCBI BLASTN to build local
alignments for two assemblies, then filtering for matches with bit scores
$>$ 200, and finally computing the fraction of bases in each assembly
with at least one alignment.  Total fractions were normalized to
self-by-self BLASTN overlap identity to account for BLAST-specific
sequence filtering.
TBLASTX comparison of the {\em Deltaproteobacteria} SAR324 sequence
was done against another assembled SAR324 sequence, acc AFIA01000002.1.

\subsection*{Compute requirement estimation}

Execution time was measured using real time from Linux bash 'time'.
Peak memory usage was estimated either by the 'memusg' script from
stackoverflow.com, peak-memory-usage-of-a-linux-unix-process, included
in the khmer repository; or by the Torque queuing system monitor, for
jobs run on MSU's HPC system.  While several different machines were
used for analyses, comparisons between unnormalized and normalized
data sets were always done on the same machine.

% Do NOT remove this, even if you are not including acknowledgments
\section*{Acknowledgments}

%This work was supported in part by the USDA (AFRI
%Competitive Grant 2010-65205-20361), NSF (DBI-0939454 and
%IOS-0923812), as well as the DOE Great Lakes Bioenergy Research Center
%(DOE BER DE-FC02-07ER64494), MSU AgBio Research, and NIH grant UH3
%DK083993.  AH was supported by NSF Postdoctoral Fellowship Award
%\#0905961.

%\section*{References}
% The bibtex filename
\bibliography{diginorm}

\section*{Figure Legends}



%%%%%%%%%%%%%%%%%%


\end{document}

